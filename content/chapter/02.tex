%!TEX root = ../../main.tex

\chapter{Beispiel Code-schnipsel einbinden}

%title wird unter dem Bsp. abgedruckt
%caption wird im Verzeichnis abgedruckt
%label wird zum referenzieren benutzt, muss einzigartig sein.

\begin{lstlisting}[caption=Code-Beispiel, label=Bsp.1]
public class HelloWorld {
	public static void main (String[] args) {
		// Ausgabe Hello World!
		System.out.println("Hello World!");
	}
}
\end{lstlisting}

%language ändert die Sprache. (Wenn nur eine Sprache verwendet wird, kann diese Sprache in development.tex geändert werden. Standardmäßig Java.)
\begin{lstlisting}[caption=Python-Code, label=Python-Code, title=Titel des Python-Codes,language=Python]
def quicksort(liste):
if len(liste) <= 1:
	return liste
pivotelement = liste.pop()
links = [element for element in liste if element < pivotelement]
rechts = [element for element in liste if element >= pivotelement]
return quicksort(links) + [pivotelement] + quicksort(rechts)
# Quelle: http://de.wikipedia.org/wiki/Python_(Programmiersprache)
\end{lstlisting}

\section{lorem ipsum}
Looking for the one superhero comic you just have to read \ac{IS}.
Following the antics and adventures of May Mayday Parker, this Spider-book has everything you could want in a comic--action, laughs, mystery and someone in a Spidey suit.
Collects Alias \#1-28, What If. Jessica Jones had Joined the Avengers \acp{IS}.
In her inaugural arc, Jessicas life immediately becomes expendable when she uncovers the potentially explosive secret of one heros true identity.

Manchmal braucht man auch Tabellen. 
Ein Beispiel sieht man in Tabelle \ref{tabelle1}, welche mit einem beliebigen Label bezeichnet werden kann. 
Die Tabelle taucht dann automatisch im Tabellenverzeichnis auf.

\begin{table}[ht!]
	\centering{

		\begin{tabular}{ | m{5cm} | m{1cm}| m{1cm} | }
			\hline
			cell1 dummy text dummy text dummy text & cell2 & cell3 \\
			\hline
			cell1 dummy text dummy text dummy text & cell5 & cell6 \\
			\hline
			cell7                                  & cell8 & cell9 \\
			\hline
		\end{tabular}
	}
	\caption{Test der Funktion der Tabelle und ihrer Darstellung}
	\label{tabelle1}
\end{table}

\section{Verweis auf Code}
Verweis auf den Code \autoref{Bsp.1}.\\
und der Python-Code \autoref{Python-Code}.

Zweite Erwähnung einer Abkürzung \ac{AGPL} (Erlärung wird nicht mehr angezeigt)