%!TEX root = ../../main.tex

\chapter{Grundlagen}
\label{chapter:2}

\section{Humanzentriertes Design (Nutzerzentrietes Design)}

User Zentriertes Design stammt vom Human Zentrierten Design ab, es wurde auf sowas wie Software abgestimmt, aber die selben prinzipien bestehen.

% TODO: die vlt einarbeiten oder sind die sachen schon drin ?
% > hier kann man dann einfach Human Centered Design erklären
% > Man könnte vlt das Prinzip auch mit dieser “Norman Door” erklären und dem leser ein wenig verinnerlichen so und das selbst so ein einfaches Prinzip von einer Türe schon so komplex sein kann. Das dass anpassen einer Webseite extrem schwer ist für den Nutzer
% Erklären was das ist was für variaten es gibt und wie das generelle vorgehen davon vlt sein kann
% It's not you. Bad doors are everywhere. - YouTube
% > Discoverability
% > Feedback

\subsection{Definition und Prinzipien des Humanzentrierten Designs}

\subsection{Anwendung von Humanzentriertem Design im Webkontext}

\section{Stand der Technik der CO2-Runter App}

\section{Webtechnologien}

\subsection{Verwendung von React und anderen Webtechnologien in der App-Entwicklung}

\subsection{Technische Grundlagen der CO2-Runter-App}
