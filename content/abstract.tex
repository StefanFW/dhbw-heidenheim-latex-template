%!TEX root = ../main.tex

\pagestyle{empty}

% override abstract headline
\renewcommand{\abstractname}{Abstract}

\begin{abstract}

    Im Zuge der Energiewende und kontinuierlicher Maßnahmen gegen den Klimawandel ist es von großer Bedeutung, das Bewusstsein der Menschen für ihren individuellen CO2-Verbrauch zu schärfen und Wege zur Reduzierung ihres CO2-Fußabdrucks aufzuzeigen. Die CO2-Runter-App, entwickelt von der Stadt Karlsruhe im Rahmen des Projekts "CO2-Runter-App", ermöglicht es den Nutzern, ihren CO2-Fußabdruck zu berechnen und mit anderen zu vergleichen. Diese Studienarbeit widmet sich der Analyse der CO2-Runter-App und der Entwicklung einer verbesserten Version, die auf die Bedürfnisse der Nutzer zugeschnitten ist. Um dieses Ziel zu erreichen, wurden Umfragen durchgeführt, um die Anforderungen der Nutzer zu ermitteln und die App entsprechend der Ergebnisse benutzerfreundlicher zu gestalten, um die aktive Teilnahme der Nutzer zu fördern. Die Ergebnisse der Umfrage werden in dieser Arbeit präsentiert, und die daraus resultierenden Änderungen an der App werden ausführlich erläutert.

\end{abstract}

\renewcommand{\abstractname}{Abstract}

\begin{abstract}

    In the context of the energy transition and ongoing measures against climate change, it is of great importance to raise awareness among people about their individual CO2 consumption and to demonstrate ways to reduce their carbon footprint. The CO2-Reduction App, developed by the city of Karlsruhe as part of the "CO2-Reduction App" project, allows users to calculate their CO2 footprint and compare it with others. This research project is dedicated to the analysis of the CO2-Reduction App and the development of an enhanced version tailored to user needs. To achieve this goal, surveys were conducted to identify user requirements and make the app more user-friendly, thus encouraging active user participation. The survey results are presented in this paper, and the resulting changes to the app are elaborated upon.

\end{abstract}